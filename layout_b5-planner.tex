% Load packages
\usepackage[letterpaper,layout=b5paper,twoside,driver=luatex,left=0.5cm, right=0.5cm, top=1.5cm, bottom=0.5cm,bindingoffset=1cm,nofoot,layouthoffset=\dimexpr(\paperwidth-\csname Gm@layoutwidth\endcsname)/2\relax,layoutvoffset=\dimexpr(\paperheight-\csname Gm@layoutheight\endcsname)/2\relax,showcrop]{geometry}
\usepackage[singlespacing]{setspace}
\usepackage{etoolbox}
\usepackage{titlecaps}
\usepackage{tabularht}
\usepackage{pgffor}

% Define counters 
\newcounter{PlannerNumber}
\setcounter{PlannerNumber}{1}
\newcounter{PlannerOverviewNumber}
\setcounter{PlannerOverviewNumber}{1}

% Define commands 
\newcommand{\SquareSpace}{\square\enspace}
\newcommand{\SquareSpaceTwo}{\square\hspace{.25em}}
\newcommand{\squareline}{\Repeat{10}{\SquareSpace}}
\newcommand{\squarelinetwo}{\Repeat{10}{\SquareSpaceTwo}}

\newcommand{\Planner}{
  \stepcounter{PlannerNumber}
  }

\newcommand{\PlannerOverview}{
  \stepcounter{PlannerOverviewNumber}
  \textsc{projeto}
  \enspace
\thePlannerOverviewNumber
  }

\newcommand{\PlannerOverviewBegin}{
  \textsc{projeto}
  \enspace
  \thePlannerOverviewNumber
}

\newcommand{\PlannerBegin}{
  \square
  \enspace
  \textsc{projeto}
  \enspace
  \thePlannerNumber
}

\newcommand{\ProgressBar}{
  \textsc{progresso}
  \hspace{1em}
  \squareline
}

\newcommand{\Repeat}[2]{%
  \foreach \n in {1,...,#1}{#2}
}

\newcommand{\PlannerOverviewSection}{%
\begin{table}[b!]
\begin{tabularx}{\textwidth}{|ll|l|}
\hline
\PlannerOverview & &  \textsc{prioridade} \phantom{mmmmmmmmmm} \\ 
\cline{3-3}
 & &  \textsc{estado} \\ \cline{3-3}
  \multicolumn{2}{|X|}{} &  \textsc{pomodori}  \\ \cline{3-3}
  \multicolumn{2}{|X|}{} &  \textsc{começo}  \\ \cline{3-3}
  \multicolumn{2}{|X|}{} &  \textsc{prazo}  \\ \hline
  \multicolumn{2}{|l|}{\ProgressBar} &  \textsc{final}  \\ 
\hline
\end{tabularx}
\end{table}
}

% Define macro to create the tables used for planning projects. 
\newcommand{\PlannerSection}{%
\begin{table}[b!]
\begin{tabularx}{\textwidth}{|lX|l|}
\hline
\textsc{descrição} & &   \textsc{prioridade} \phantom{mmmmmmmmmm} \\ \cline{3-3}
& &  \textsc{estado} \\ \cline{3-3}
 & &  \textsc{pomodori} \\ \cline{3-3}
 & &  \textsc{começo} \\ \cline{3-3}
 & &  \textsc{prazo} \\
 \hline
\multicolumn{2}{|l|}{\ProgressBar} &  \textsc{final} \\
\hline
\end{tabularx}
\end{table}

\begin{table}[b!]
\begin{tabularx}{\textwidth}{|c|c|Y|c|c|c|}
\hline
&  \textsc{p} & \textsc{tarefas}   & \textsc{e} & \textsc{final} & \textsc{prazo} \\ \hline
1 & &  &   & &   \\ \hline
2 & &  &   & &  \\ \hline
3 & &  &   & &  \\ \hline
4 & &  &   & &  \\ \hline
5 & &  &   & &  \\ \hline
6 &  & &   & &  \\ \hline
7 &  & &   & &  \\ \hline
8 &  & &   & &  \\ \hline
9 &  & &   & &  \\ \hline
10 &   & &  & &  \\ \hline
11 &    & &  & &  \\ \hline
12 &   & &  & &  \\ \hline
13 &   &  & & &  \\ \hline
14 & &   &  & &  \\ \hline
15 & &   &  & &  \\ \hline
16 & &   &  & &  \\ \hline
17  & &  &  & &  \\ \hline
18 & &  &  &  & \\ \hline
19 & &  &  & &  \\ \hline
20 & &  &  & &  \\ \hline
21 & &  &  & &  \\ \hline
22 & &  &  & &  \\ \hline
23 & &  &  & &  \\ \hline
24 & &  &  & &  \\ \hline
25 & &  &  & &  \\ \hline
\end{tabularx}
\end{table}

\begin{table}[b!]
    \begin{minipage}{.5\linewidth}
      \centering
        \begin{tabularx}{0.99\textwidth}{|Y|}
\hline
\textsc{avaliação} \\ 
 \\
 \\
 \\
 \\
 \\
 \\
 \\
 \\
 \\
 \\
 \\
 \\
 \\ \hline
        \end{tabularx}
    \end{minipage}%
    \begin{minipage}{.5\linewidth}
      \centering
        \begin{tabularx}{0.99\textwidth}{|Y|}
\hline
\textsc{ideias} \\
 \\
 \\
 \\
 \\
 \\
 \\
 \\
 \\
 \\
 \\
 \\
 \\
 \\ \hline
        \end{tabularx}
    \end{minipage} 
\end{table}
}

% Define macro to create a tikz picture with lines (as in a notebook).
\newcommand{\WritingLines}{%
\begin{figure}[b!]
  \parbox[t][\paperheight]{\paperwidth}{
    \begin{tikzpicture}[overlay]
      \foreach \i in {-0.31, -0.91, ..., -23.5}{\draw [gray, thin]
        (-3cm,\i) -- (17cm,\i);}
    \end{tikzpicture}}
\end{figure}
}

% Define the some macros to have bigger fonts in titles.
\newcommand\hugetitle{\fontsize{55}{0}\selectfont}
\newcommand\largetitle{\fontsize{30}{0}\selectfont}
\newcommand\hugenumber{\fontsize{150}{160}\selectfont}
\newcommand\largenumber{\fontsize{40}{0}\selectfont}
\newcommand\hugechapter{\fontsize{70}{80}\selectfont}
\newcommand\largechapter{\fontsize{25}{30}\selectfont}


% Define the '\largechapter macro. 
\renewcommand{\thechapter}{\Roman{chapter}}

% Chage the look of chapter pages using titlesec.
\usepackage{titlesec}
\titleformat{\chapter}[display] {\centering}
{}
{1em}
{\hugenumber\thechapter\linebreak 
\includegraphics[width=0.9\textwidth]{swirls_ornament}
\mychapterfamily\hugechapter}
[]

% Chage the look of chapter pages using titlesec.
\patchcmd{\chapter}{plain}{empty}{}{}
\patchcmd{\part}{plain}{empty}{}{}

\usepackage[extramarks]{titleps}

\newmarkset{PlannerNumber}

\newcommand\PlannerNumber[1]{%
\clearpage
\stepcounter{PlannerNumber}
 #1
}

\newpagestyle{main}{
\sethead[\huge\rmfamily\titlecap{\chaptertitle}\enspace\thechapter][][\Large\the\year] % even
{}{}{\LARGE\thepage}} % odd

\newpagestyle{planner}{
\sethead[\huge\rmfamily Projeto \enspace\thePlannerNumber][][\Large\the\year] % even
{}{}{\LARGE\thepage}} % odd

% Set planner as the default page style. 
\pagestyle{planner}

% Don't number parts, chapter, sections, and subsections.
\setcounter{secnumdepth}{0}
