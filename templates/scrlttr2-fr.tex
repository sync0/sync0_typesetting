\input{/home/sync0/Documents/typography/scrlttr2_french.tex}
\setkomavar{subject}{Objet : Candidature au poste d'assistant
  diplômé en histoire de la pensée politique.}
\begin{document}
\begin{letter}{\textbf{Institut d'études politiques}, \\
    Centre Walras Pareto,\\
    Université de Lausanne,\\
    CH-1015 Lausanne,\\
    Vaud, Suisse}

  \opening{Mesdames, Messieurs,}

  Désireux de faire une thèse sur la pensée socialiste et le monde
  du travail lors des controverses de 1848 en France sous la
  direction du professeur Thomas Bouchet, j'ai l'honneur de
  proposer ma candidature au poste d'assistant diplômé en histoire
  de la pensée politique, souhaitant m'intégrer à l'Institut
  d'Études Politiques et au Centre Walras Pareto de l'Université
  de Lausanne.

  Titulaire du Master 2 \textit{Économie en sciences humaines},
  obtenu avec mention \textit{très bien} à l'Université Paris 1
  Panthéon-Sorbonne, je souhaite, en empruntant la voie du
  doctorat, approfondir mes connaissance en histoire
  intellectuelle, sociologie historique et études sociales de la
  science (\textit{Science Studies}) pour devenir un chercheur en
  histoire de la pensée politique et économique.

  Depuis mes études de licence, je m'intéresse à l'histoire
  sociale, l'histoire économique et l'histoire des idées, comme en
  témoignent mes nombreux cours en histoire et culture coréenne,
  le cours en histoire des théories du développement, et les
  divers cours en théorie économique. Passionné par les travaux de
  Philip Mirowski sur l'emploi de métaphores avec les sciences
  «dures » dans la science économique, j'ai décidé de faire un
  Master à l'Université Paris 1, où j'ai écrit un mémoire sur le
  rapport entre les origines intellectuelles du concept de la
  «division du travail mental», la pensée managériale de Charles
  Babbage et la réorganisation du travail dans les ateliers
  londoniens que Babbage a utilisé pour produire ses machines à
  calculer.

  C'est, donc, dans la continuité de mes intérêts de recherche que
  s’insère mon projet de thèse, dans lequel je voudrais développer
  un «regard des regards» à propos des controverses sur des
  différents aspect du monde du travail (son organisation, sa
  représentation politique, son encadrement juridique) autour de
  l'année 1848 en France, tout en analysant comment ces débats ont
  façonné la pensée socialiste.

  Au-delà de ce projet, je voudrais dédier ma recherche à
  l'histoire des concepts---le \textit{travail},
  l'\textit{emploi}, le \textit{développement}, la
  \textit{gouvernance} et la \textit{monnaie}---tel qu'ils
  apparaissent dans les discours politiques, économiques,
  scientifiques et juridiques au \sie{19} et au \sie{20} siècle.
  Spécifiquement, je m'intéresse aux processus historiques qui
  résultent dans leur adoption comme des formes consensuelles de
  découper la «réalité» sociale et de comprendre les changements
  de celle-ci. Compte tenu du fait qu'un tel «programme » requiert
  l'adoption d'une épistémologie historique critique et le
  dépassement des préoccupations disciplinaires étroites, c'est,
  donc, avec toute naturalité que je souhaite intégrer le Centre
  Walras Pareto. Étant un des rares centres de recherche dédiés à
  l'étude historique des idées économiques et politiques, je suis
  convaincu que les conditions offertes par votre école (accès aux
  ressources, séminaires, encadrement) me permettront d'appliquer
  mes compétences historiographiques et d'en acquérir nouvelles
  pour mener à bien mon projet de thèse.

  En candidatant pour ce poste d'assistant diplômé, je voudrais
  vous apporter mon assiduité et mon ouverture d'esprit pour
  travailler avec d'autres chercheurs : En outre du professeur
  Bouchet, je serais très ravi de collaborer avec Harro Maas et
  Roberto Baranzini. Ma formation interdisciplinaire et mes
  connaissances linguistiques---qui me permettent de lire en
  français, anglais, espagnol, et portugais---seront des atouts
  importants pour soutenir les tâches d'enseignement et la
  rédaction d'articles.

  Dans l'attente de vous rencontrer pour développer mes
  motivations et vous apporter de plus amples renseignements, je
  vous prie de croire, Mesdames, Messieurs, à l'expression de mes
  salutations respectueuses.
  
  \closing{Bien cordialement,}
  \includegraphics[scale=1]{signature.pdf}\\ 
  % \includegraphics[scale=0.5]{minjung_signature.pdf}\\ 
  % 
  % \ps \textsc{PS}: Veuillez trouver en pièce jointe le \textsc{rib} du
  % compte \textsc{bnp} Paribas.
  % 
  % \encl{\textsc{rib} du compte \textsc{bnp} Paribas.}
  % \encl{Photocopy of something interesting\\
  % Photocopy of something rather dull}
  % 
\end{letter}
\end{document}

%%% Local Variables:
%%% mode: latex
%%% TeX-engine: luatex 
%%% TeX-master: t
%%% End:
